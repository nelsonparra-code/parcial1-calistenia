\documentclass{article}
\usepackage[utf8]{inputenc}
\usepackage[spanish]{babel}
\usepackage{listings}
\usepackage{graphicx}
\graphicspath{ {images/} }
\usepackage{cite}

\begin{document}

\begin{titlepage}
    \begin{center}
        \vspace*{1cm}
            
        \Huge
        \textbf{Parcial 1 (15\%)}
            
        \vspace{0.5cm}
        \LARGE
        Calistenia
            
        \vspace{1.5cm}
            
        \textbf{Nelson Fernando Parra Guardia}
            
        \vfill
            
        \vspace{0.8cm}
            
        \Large
        Despartamento de Ingeniería Electrónica y Telecomunicaciones\\
        Universidad de Antioquia\\
        Medellín\\
        Marzo de 2021
            
    \end{center}
\end{titlepage}

\newpage

Seleccionar su mano mas habil y realizar los siguientes pasos con esa unica mano:
\begin{enumerate}

    \item Levantar la hoja conformando su forma inicial.
    \item Ubicar la hoja de manera horizontal sobre la superficie plana mas cercana a las tarjetas, evitando que la hoja permanezca encima de las cartas nuevamente, y soltar.
    \item Tomar ambas tarjetas (conservando su forma) y cambiarlas a posicion vertical (si no lo estan) reposando sobre la superficie donde se encuentran.
    \item Posicionar el dedo pulgar y el dedo indice simultaneamente sobre ambas esquinas superiores de ambas tarjetas.
    \item En esa posicion, agarrar las tarjetas y mantenerlas en posicion vertical (formando un angulo recto con respecto a la superficie donde se ubicaban).
    \item Sin soltarlas y manteniendo su orientacion (vertical con respecto a la superficie plana), hacer que las tarjetas esten en contacto con la hoja de papel previamente reubicada.
    \item Sin soltar en ningun momento las tarjetas, y con el dedo anular tocar la tarjeta mas proxima a usted y con el dedo medio tocar la tarjeta mas alejada a usted, empezar a separar lentamente ambas tarjetas para ir formando una piramide.
    \item Cuando considere que ambas tarjetas pueden permanecer en posicion de piramide (sin la ayuda de su mano), empiece lentamente a retirar su mano de las tarjetas, hasta soltarlas completamente.
    \item Finalmente, si no permanecen en la posicion de piramide deseada:
    \begin{enumerate}
    
        \item Ubicar las tarjetas en su forma inicial (una tarjeta por encima de la otra, en posicion vertical sobre la superficie plana).
        \item Repetir las instrucciones desde el paso 4.
        
    \end{enumerate}
    
\end{enumerate}

\end{document}
